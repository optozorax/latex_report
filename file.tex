\mytitlepage{прогресса}{1}{Создание отчетов}{Создание отчета в LaTeX}{ПМ-63}{Шепрут И.И.}{1}{Шепрут И.И.}{2018}

\section{Цель работы}

Показать пример использования LaTeX для написания отчетов.

\section{Анализ задачи}

Для этого необходимо собрать минимальное необходимое количество примеров, а так же оставить ссылки на документацию.

\section{Код программы}

Для демонстрации этого написан текущий файл с таким кодом:

\mycodeinput{latex}{my_title.tex}{title.tex}
\mycodeinput{latex}{file.tex}{file.tex}

\section{Таблица}

\tabulinesep=0.3mm
\noindent\scriptsize{\texttt{\begin{tabu}{|X[-1,c]||X[1,c]|X[1,c]|X[1,c]|X[1,c]|X[1,c]|}
\hline
	\# & Название & Документация & Пример & PDF & Присутствует в этом файле \\
\hline
\hline
	1 & Code & + & + & + & + \\
\hline
	2 & Graph & + & + & + & + \\
\hline
	3 & Graph by data & + & + & + & $\pm$ \\
\hline
	4 & List & + & + & + & + \\
\hline
	5 & Table & + & + & + & + \\
\hline
\end{tabu}}}

\section{График}

\noindent\begin{tikzpicture}
	\begin{axis}[
		xlabel=Пример,
		ylabel=Присутствует в этом файле?,
		width=\textwidth, 
		height=5cm]
	\addplot[color=red, mark=x] coordinates {
		(1,1)
		(2,1)
		(3,0.5)
		(4,1)
		(5,1)
	};
	\end{axis}
\end{tikzpicture}

\section{Пример формулы}

$$ f(x) = \sum_{k=0}^n \frac{f^{(k)}(x_0)}{k!}(x-x_0)^k + \underbrace{\frac{1}{n!}\int\limits_{x_0}^x f^{(n+1)}(t)(x-t)^n dt}_{\text{интегральный остаточный член}} $$

Код \mycodeinline{latex}{$\frac{1}{a}$} сделает дробь $\frac{1}{a}$.

\section{Выводы}

\noindent\normalsize{\begin{easylist}
\ListProperties(Hang1=true)
& Было показано множество полезных примеров.
& Документация расположена в папках примеров.
& В ворде некоторые вещи делаются в миллион раз проще и в пару кликов.
& Зато в латехе ничего не съедет, и можно автоматически генерировать таблицы.
\end{easylist}}