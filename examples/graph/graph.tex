\documentclass[a4paper,12pt]{article}
\usepackage[russian]{babel}

\usepackage{amssymb,amsmath}
\parindent=24pt
\parskip=0pt
\tolerance=2000

\usepackage{geometry}
\geometry{
	a4paper,
	total={170mm,257mm},
	left=20mm,
	top=20mm,
}

% Для контура вокруг текста
\usepackage[outline]{contour}

% Пакеты от tikz
\usepackage{tikz}
\usepackage{graphics}
\usepackage{pgfplots}
\usepackage{pgfplotstable}
\usepackage{xcolor}
\usetikzlibrary{calc}
\usetikzlibrary{through}
\usetikzlibrary{intersections}
\usetikzlibrary{patterns}
\usetikzlibrary{scopes}
\usetikzlibrary{decorations.pathreplacing}
\usetikzlibrary{arrows.meta}

\begin{document}


\begin{tikzpicture}
\begin{axis}
\addplot[color=red]{exp(x)};
\end{axis}
\end{tikzpicture}

\begin{tikzpicture}
\begin{axis}
\addplot3[
    surf,
]
{exp(-x^2-y^2)*x};
\end{axis}
\end{tikzpicture}

\begin{tikzpicture}
\begin{axis}[
    title=Example using the mesh parameter,
    hide axis,
    colormap/cool,
]
\addplot3[
    mesh,
    samples=50,
    domain=-8:8,
]
{sin(deg(sqrt(x^2+y^2)))/sqrt(x^2+y^2)};
\addlegendentry{$\frac{sin(r)}{r}$}
\end{axis}
\end{tikzpicture}

\begin{tikzpicture}
\begin{axis}[
    axis lines = left,
    xlabel = $x$,
    ylabel = {$f(x)$},
]
% Below the red parabola is defined
\addplot [
    domain=-10:10, 
    samples=100, 
    color=red,
]
{x^2 - 2*x - 1};
\addlegendentry{$x^2 - 2x - 1$}
% Here the blue parabloa is defined
\addplot [
    domain=-10:10, 
    samples=100, 
    color=blue,
    ]
    {x^2 + 2*x + 1};
\addlegendentry{$x^2 + 2x + 1$}
 
\end{axis}
\end{tikzpicture}

% Задаем толщину контура
\contourlength{0.7pt}

\begin{tikzpicture}

	% Grid
	\draw[step=0.5, very thin, gray] (-0.5,-1) grid ++(5, 2);

	% Graphs
	\draw[domain=-0.5:4.5,smooth,samples=500,variable=\x,black, very thick] plot ({\x},{cos(deg(\x/4*2*pi))});

	% Axes
	\draw[->, thick] (-0.7, 0) -- (4.7, 0);
	\draw[->, thick] (0, -1.2) -- (0, 1.2);

	\draw[-, dashed] (2, 0) -- (2, -1);
	\draw[-, dashed] (4, 0) -- (4, 1);

	% Nodes
	\draw (4.9,0) node {$x$};
	\draw (0, 1.4) node {$y$};

	\draw (-0.3, -0.2) node {\contour{white}{$0$}};
	\draw (-0.3, 1) node {\contour{white}{$1$}};

	\draw (1, -0.3) node {\contour{white}{$\frac{\pi}{2}$}};
	\draw (2, -0.3) node {\contour{white}{$\pi$}};
	\draw (3, -0.3) node {\contour{white}{$\frac{3\pi}{2}$}};
	\draw (4, -0.3) node {\contour{white}{$2\pi$}};

	\draw (2, 0.5) node[scale=1.3] {\contour{white}{$y=\cos x$}};

	% Points
	\fill [black] (0, 0) circle (1.5pt);
	\fill [black] (0, 1) circle (1.5pt);
	\fill [black] (1, 0) circle (1.5pt);
	\fill [black] (2, 0) circle (1.5pt);
	\fill [black] (3, 0) circle (1.5pt);
	\fill [black] (4, 0) circle (1.5pt);

\end{tikzpicture}

\begin{tikzpicture}
	% Координаты всех необходимых точек
	\coordinate (M) at (0.000,0.000);
	\coordinate (B) at (-5.000,-4.000);
	\coordinate (C) at (5.000,-4.000);
	\coordinate (BN) at (-5.000,-4.100);
	\coordinate (BU) at (-5.000,-3.900);
	\coordinate (CN) at (5.000,-4.100);
	\coordinate (CU) at (5.000,-3.900);
	\coordinate (I1) at (-3.500,-4.000);
	\coordinate (I2) at (3.500,-4.000);
	\coordinate (J1) at (-3.000,-4.000);
	\coordinate (J2) at (3.000,-4.000);
	\coordinate (IN1) at (-3.500,-4.100);
	\coordinate (IU1) at (-3.500,-3.900);
	\coordinate (JN1) at (-3.000,-4.100);
	\coordinate (JU1) at (-3.000,-3.900);
	\coordinate (IN2) at (3.500,-4.100);
	\coordinate (IU2) at (3.500,-3.900);
	\coordinate (JN2) at (3.000,-4.100);
	\coordinate (JU2) at (3.000,-3.900);
	\coordinate (K1) at (-3.250,-4.000);
	\coordinate (K2) at (3.250,-4.000);
	\coordinate (F1) at (-2.112,-2.600);
	\coordinate (F2) at (2.112,-2.600);
	\coordinate (FX1) at (-2.112,-4.000);
	\coordinate (FY1) at (-3.250,-2.600);
	\coordinate (FX2) at (2.112,-4.000);
	\coordinate (FY2) at (3.250,-2.600);
	\coordinate (P1) at (-1.138,-1.400);
	\coordinate (P2) at (1.138,-1.400);
	\coordinate (PX1) at (-1.138,0.000);
	\coordinate (PY1) at (0.000,-1.400);
	\coordinate (PX2) at (1.138,0.000);
	\coordinate (PY2) at (0.000,-1.400);

	% Материальная точка
	\fill [black] (M) circle (3.5pt);
	
	% Стержень
	\draw[-,line width=2pt,color=gray!40] (B) -- (C);
	\draw[-,color=gray!40,line width=1pt] (BN) -- (BU);
	\draw[-,color=gray!40,line width=1pt] (CN) -- (CU);
	
	% Кусочки стержня
	\draw[-,line width=2pt] (I1) -- (J1);
	\draw[-,line width=2pt] (I2) -- (J2);
	\draw[-,line width=1pt] (IN1) -- (IU1);
	\draw[-,line width=1pt] (JN1) -- (JU1);
	\draw[-,line width=1pt] (IN2) -- (IU2);
	\draw[-,line width=1pt] (JN2) -- (JU2);
	
	% Пунктирные линии для сил
	\draw[-,dashed,very thin] (F1) -- (P1);
	\draw[-,dashed,very thin] (F2) -- (P2);
	\draw[-,dashed,very thin] (F1) -- (FX1);
	\draw[-,dashed,very thin] (F2) -- (FX2);
	\draw[-,dashed,very thin] (F1) -- (FY1);
	\draw[-,dashed,very thin] (F2) -- (FY2);

	\draw[-,dashed,very thin] (P1) -- (PX1);
	\draw[-,dashed,very thin] (P2) -- (PX2);
	\draw[-,dashed,very thin] (P1) -- (PY1);
	\draw[-,dashed,very thin] (P2) -- (PY2);
	
	% Силы
	\draw[>=latex,->,very thin] (K1) -- (F1);
	\draw[>=latex,->,very thin] (K2) -- (F2);
	\draw[>=latex,->,very thin] (K1) -- (FX1);
	\draw[>=latex,->,very thin] (K2) -- (FX2);
	\draw[>=latex,->,very thin] (K1) -- (FY1);
	\draw[>=latex,->,very thin] (K2) -- (FY2);
	
	\draw[>=latex,->,very thin] (M) -- (P1);
	\draw[>=latex,->,very thin] (M) -- (P2);
	\draw[>=latex,->,very thin] (M) -- (PX1);
	\draw[>=latex,->,very thin] (M) -- (PX2);
	\draw[>=latex,->,very thin] (M) -- (PY1);
	\draw[>=latex,->,very thin] (M) -- (PY2);
	
	% Подписи сил
	\node[yshift=12pt] at (PY2) {\contour{white}{$2\vec F_{iy}$}};
	\node[yshift=8pt,xshift=-8pt] at (F2) {\contour{white}{$\vec F_{n-i}$}};
	\node[yshift=8pt,xshift=8pt] at (F1) {\contour{white}{$\vec F_{i}$}};
	\node[yshift=12pt] at (PX2) {\contour{white}{$\vec F_{(n-i)x}$}};
	\node[yshift=12pt] at (PX1) {\contour{white}{$\vec F_{ix}$}};
	\node[yshift=12pt] at (4,-2) {$ \boxed{\vec F_{ix} + \vec F_{(n-i)x} = \vec 0} $};
	\node[yshift=12pt] at (-4,-2) {$ \boxed{\vec F_{iy} + \vec F_{(n-i)y} = 2\vec F_{iy}} $};
	
	% Подписи расстояний
	\draw [decorate,decoration={brace,amplitude=5pt,mirror,raise=2pt}] (K1) -- ($ (B)!.5!(C) $) node [black,midway,yshift=-15pt] {\contour{white}{$l_i$}};
	\draw [decorate,decoration={brace,amplitude=5pt,mirror,raise=2pt}] ($ (B)!.5!(C) $) -- (K2) node [black,midway,yshift=-15pt] {\contour{white}{$l_{n-i}$}};
	\draw[-,dashed,very thin] (PY2) -- ($ (B)!.5!(C) $);
	\node at ($ (PY2)!.5!($ (B)!.5!(C) $) $) {\contour{white}{$a$}};
	\node[yshift=-25pt] at ($ (B)!.5!(C) $) {$\boxed{l_i=l_{n-i}}$};
\end{tikzpicture}

\end{document}